\lhead{\emph{Implementation}}
\chapter{Implementation}

\section{Headset}

\section{Head Gestures Recognition}

\subsection{Dynamic Time Warping}

DTW - Dynamic Time Warping 

Dynamic Time Warping \cite{salvador_toward_2007}

Accelerometer-based DTW \cite{akl_accelerometer-based_2010}

\subsection{Sensor data}
Gyroscope data, 1 observation, 1 sequence

\subsection{Removing noise}
Start and end noise removal

\section{iOS Application}
The platform used in this project is Apples iOS. The main reason for using this platform is that when the project started the Intelligent Headset only provided an SDK compatible with iOS. The iOS is a mature OS currently at version 7.1. The Spatial Music Menu are running version 6+ and optimized for the iPad.

As an overview and introduction of the iOS framework is out of this thesis scope - the interested reader is referred to Apples developer portal \cite{apple_apple_2014} where a comprehensive documentation on the framework is available. Instead this section will focus on the application architecture, the design patterns and most important functionality of the Spatial Music Menu.

\subsection{Overview}
An overview of the application architecture is shown in figure \ref{fig:apparchitecture}.

\begin{figure}[htbp]
	\centering
		\includegraphics[width=\textwidth,height=\textheight,keepaspectratio]{./Figures/app-architecture.pdf}
		\rule{35em}{1pt}
	\caption[App architecture]{iOS app architecture overview}
	\label{fig:apparchitecture}
\end{figure}



% Milestones - POC
% - Create and setup xcode project including lib references for Intelligent Headset and Spotify
% - Play a users Spotify songs and be able to use panning and volume
% - Place multiple songs in the horizontal range 0-180 degrees of the user using IHS rotation
% - Implement nodding and shaking feedback

% Further improvement
% - Exploring the Spotify lib/playlists
% - Activation of head gestures
% - Fine tuning

% Learnings
% Streaming from Spotify to OpenAL could be difficult. It requires decompressing of the stream; 1) maybe not possible with libspotify, 2) If so very heavy - could affect user experience, 3) OpenAL restrictions
% Idea (avoiding OpenAL's limited audio input properties):
% As we only need horizontal 180 degrees we could use panning. Place x songs from 0-180 degrees. As the user rotates head a song pans in from the direction with increasing volume.