\chapter{Implementation}

[TODO: Overview Image (headset, deezer, ipad connections)]




\section{Application}

An overview of the application architecture is shown in figure \ref{fig:apparchitecture}.

\begin{figure}[htbp]
	\centering
		\includegraphics[width=\textwidth,height=\textheight,keepaspectratio]{./Figures/app-architecture.pdf}
		\rule{35em}{1pt}
	\caption[App architecture]{iOS app architecture overview}
	\label{fig:apparchitecture}
\end{figure}

\section{Intelligent Headset}
[TODO]


\section{Head gesture recognition}
DTW - Dynamic Time Warping 

Dynamic Time Warping \cite{salvador_toward_2007}

Accelerometer-based DTW \cite{akl_accelerometer-based_2010}


\section{iOS application}
System design overview...

% Milestones - POC
% - Create and setup xcode project including lib references for Intelligent Headset and Spotify
% - Play a users Spotify songs and be able to use panning and volume
% - Place multiple songs in the horizontal range 0-180 degrees of the user using IHS rotation
% - Implement nodding and shaking feedback

% Further improvement
% - Exploring the Spotify lib/playlists
% - Activation of head gestures
% - Fine tuning

% Learnings
% Streaming from Spotify to OpenAL could be difficult. It requires decompressing of the stream; 1) maybe not possible with libspotify, 2) If so very heavy - could affect user experience, 3) OpenAL restrictions
% Idea (avoiding OpenAL's limited audio input properties):
% As we only need horizontal 180 degrees we could use panning. Place x songs from 0-180 degrees. As the user rotates head a song pans in from the direction with increasing volume.