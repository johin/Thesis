\chapter{Evaluation}

\section{Controlled environment}

\section{Ecological evaluation}

% Iterations, measurable comparison between new system and traditional

% 2 evaluations - closed lab (1 day) and open (real life, week(s))

% Idea for closed lab exercise - Multiple lists of songs. A user shouls navigate and play the different songs with head gestures and normal navigation. Compare these in relation to time taken, cognitive load (eyes and at least one hand occupied), user feel of frustration (cognitive load) when navigating

% Final evaluation:
% Idea: Time to find a song, level of frustration (cognitive load) for finding song

% NB: For final evaluation - device with 3G+ connection and added to apple developer team, should be executed latest mid of April so finished end of April (1/2 weeks trial), 2 testpersons -> 1 experienced tech person and 1 non-tech/average user


% Maybe comfort as a measurement?
%(Taken from Brewster article)
%The final measure taken was comfort. This was based around a new scale developed by Knight et al. [10] called the Comfort Rating Scale (CRS) which assesses various aspects to do with the comfort of a wearable device. For a device to be accepted and used it needs to be comfortable and people need to be happy to wear it. Using a range of 20- point rating scales similar to NASA TLX, CRS breaks com- fort into 6 categories: emotion, attachment, harm, perceived change, movement and anxiety. Knight et al. have used it to assess the comfort of two wearable devices they are building in their research group. Using this will allow us to find out more about the actual acceptability our systems.