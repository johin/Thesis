\lhead{\emph{Evaluation}}
\chapter{Evaluation}
TODO: introduction, hypothesis/questions?

Limitations and scope\\
Biased participants (friends) want to perform good

\section{Method}
% Intro, overall
The user were to perform tasks while riding on a stationary bike. To simulate obstacles that the user should be aware of or respond to in a real world biking scenario, a screen was placed in front of the user displaying circles in a random time interval. The user should detect these circles by clicking on a button attached close to the user preferred hand on the steer (in this case an usb mouse strapped with tape).

% Task description
Artist - Album - Track\\
Task traditional: Pick phone from pocket, activate, navigate, deactivate, back in pocket

% Prerequisites
Before starting the experiment the user were instructed in how the Spatial Music Menu works i.e. which head gestures to use for navigating and how the menu structure looks. They then got a chance to try out the system both standing still and while riding the stationary bike. In the standing still scenario the user was allowed to look at the menu envisioned on the iPad screen to get a sense of the menu structure and interaction feedback. Besides trying out the Spatial Music Menu the user also got to try out a traditional music player (Deezer, TODO: ref)

\subsection{Logging}

\subsection{Observing}
Video recording to count gesture mismatches

\subsection{Questionaire}
NASA [TLX]


\section{Participants}
...


\section{Results}
...

\subsection{User performance}
...

\subsection{Questionaire results}


%Track:\\
%Quadratic track marked with cones including a cone in the middle on every side for the user to navigate around (track 50x50m)

%User/device setup:\\
%1. User head gestures setup and tracks setup\\
%2. User demoing both systems while standing still (practice)\\
%3. User demoing both systems while biking (practice)

%Round:\\
%1. Before start user get task - a track in which he/she should navigate to\\
%2. User starts biking - when sign given (whistle), the user should perform the task\\
%3. When task is finished the user raises a hand and stops\\
%4. Step 1-3 is repeated 5 times\\
%5. This evaluation is performed with both traditional and prototype system (3 times for each)

%Measurements:\\
%1. Time taken to complete a task (every task is noted on paper)\\
%2. Distance in total for 1 round\\
%3. Prototype system logging of activities (e.g. menu state switch, device connection state, gestures recognized)

%Results linked to problem statement:\\
%1. [Efficiency] Time and distance comparison between traditional and prototype system\\
%2. [Learnability] Comparison/progress between 1. round and 3. round with new prototype system\\
%3. [Usability (cognitive load)] Users answering questions (form)\\
%4. [Suitability to real-world hands- and eyes occupied situations] Summary of 1-3 above







% Iterations, measurable comparison between new system and traditional

% 2 evaluations - closed lab (1 day) and open (real life, week(s))

% Idea for closed lab exercise - Multiple lists of songs. A user shouls navigate and play the different songs with head gestures and normal navigation. Compare these in relation to time taken, cognitive load (eyes and at least one hand occupied), user feel of frustration (cognitive load) when navigating

% Final evaluation:
% Idea: Time to find a song, level of frustration (cognitive load) for finding song

% NB: For final evaluation - device with 3G+ connection and added to apple developer team, should be executed latest mid of April so finished end of April (1/2 weeks trial), 2 testpersons -> 1 experienced tech person and 1 non-tech/average user


% Maybe comfort as a measurement?
%(Taken from Brewster article)
%The final measure taken was comfort. This was based around a new scale developed by Knight et al. [10] called the Comfort Rating Scale (CRS) which assesses various aspects to do with the comfort of a wearable device. For a device to be accepted and used it needs to be comfortable and people need to be happy to wear it. Using a range of 20- point rating scales similar to NASA TLX, CRS breaks com- fort into 6 categories: emotion, attachment, harm, perceived change, movement and anxiety. Knight et al. have used it to assess the comfort of two wearable devices they are building in their research group. Using this will allow us to find out more about the actual acceptability our systems.